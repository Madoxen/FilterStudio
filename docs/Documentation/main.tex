\documentclass{article}
\usepackage[utf8]{inputenc}
\usepackage{polski}
\usepackage{multirow}
\usepackage[margin=2.5cm]{geometry}



\title{Programowanie Obiektowe i Graficzne\\
Projekt zespołowy\\
\textbf{FilterStudio}}
\author{Skład zespołu projektowego: \\ Paweł Chłąd\\Daniel Jambor\\ Bartek Meller }

\begin{document}

\maketitle

\begin{center}
    \begin{tabular}{ |c|c|c| }
        \hline
        Imię Nazwisko                  & Odpowiedzialny za                        & Zrealizował zadania                      \\ \hline
        \multirow{3}{*}{Paweł Chłąd}   &                                          & 1A. Rozplanowanie architektury           \\
                                       & Architekurę aplikacji                    & 1B. Implementacja architektury aplikacji \\
                                       &                                          & 1C. Implementacja filtrów konwolucyjnych \\ \hline
        \multirow{2}{*}{Daniel Jambor} & UI/UX                                    & 2A. Projektowanie UI/UX                  \\
                                       &                                          & 2B. Wdrożenie założeń UI/UX              \\ \hline
        Bartek Meller                  & Walidacja danych i dodatkowe elementy UI & 3A. Walidacja danych na textBoxach       \\ \hline
    \end{tabular}
\end{center}

\pagebreak


\section{Opis projektu}
FilterStudio to aplikacja pozwalająca na budowanie systemów filtrów do zdjęć. Najczęściej są to filtry konwolucyjne, ale architektura aplikacji pozwala
na dołączanie innych rodzajów filtrów.
\section{Wymagania}
Projekt spełnia następujące założenia funkcjonalne:
\begin{itemize}
    \item Użytkownik może budować filtry konwolucyjne
    \item Użytkownik może składać wiele filtrów w jeden \textit{projekt}
    \item Użytkownik może zapisać projekt, wraz z konfiguracjami filtrów
    \item Użytkownik może wprowadzić obraz do programu aby poddać go obróbce filtrom
    \item Użytkownik może zapisać przefiltrowany obraz 
\end{itemize}

Założenia niefunkcjonalne są następujące:
\begin{itemize}
    \item Solidna architektura aplikacji pozwalająca na łatwe wprowadzanie nowych filtrów do programu
    \item Szybkie wykonywanie filtrów konwolucyjnych poprzez użycie surowego dostępu do danych bitmapy
\end{itemize}


\section{Przebieg realizacji}
Każdy z wykonawców opisuje wykonane przez siebie zadania. Należy zamieścić ewentualnie schemat bazy danych (gdy aplikacja jest bazodanowa), opis algorytmu, gdy aplikacja jest związana z SSI lub inną algorytmiką. W przypadku zastosowania wzorców projektowych proszę to tu podkreślić. Należy też zwrócić uwagę na architekturę aplikacji tzn. jeśli używacie MVVM to podkreślamy tu to.
W miejscu tym należy opisać interfejs publiczny stworzonych w ramach projektu klas. Mile widziane diagramy uml.
\section{Instrukcja użytkownika}
Opis działania stworzonej aplikacji ze zrzutami ekranów ilustrujące sposób działania aplikacji.
\section{Podsumowanie i wnioski}
W miejscu tym piszemy co zrealizowaliśmy, z czym były problemy. Jakie są dalsze kierunki rozwoju tej aplikacji.  
\section{Dodatek - udokumentowanie wykorzystania systemu kontroli wersji}
(może być zrzut/zrzuty ekranu pokazujące, że do realizacji użyliście Państwo systemu kontroli wersji  - wymaga tego efekt E7)

Uwaga - do dokumentacji proszę nie wklejać całego kodu aplikacji.  W sekcji realizacja można zmieścić fragmenty kodu, jeśli chcecie zwrócić uwagę na coś co było bardzo wymagające i konieczne jest dogłębnego jego omówienia. 
Poza tym proszę komentować kod aplikacji - to jest istotna część dokumentacji projektu.



\end{document}